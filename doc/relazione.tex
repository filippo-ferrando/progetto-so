\documentclass{article}

\usepackage[italian]{babel}
\usepackage{listings}
\usepackage[letterpaper,top=1cm,bottom=2cm,left=3cm,right=3cm,marginparwidth=1.75cm]{geometry}
\usepackage{xcolor}

% Useful package
\usepackage{amsmath}
\usepackage{graphicx}
\usepackage[colorlinks=true, allcolors=blue]{hyperref}

\definecolor{codegreen}{rgb}{0,0.6,0}
\definecolor{codegray}{rgb}{0.5,0.5,0.5}
\definecolor{codepurple}{rgb}{0.90,0,0.82}
\definecolor{backcolour}{rgb}{0.95,0.95,0.92}

\lstdefinestyle{mystyle}{
    backgroundcolor=\color{backcolour},   
    commentstyle=\color{codegreen},
    keywordstyle=\color{magenta},
    numberstyle=\tiny\color{codegray},
    stringstyle=\color{codepurple},
    basicstyle=\ttfamily\footnotesize,
    breakatwhitespace=false,         
    breaklines=true,                 
    captionpos=b,                    
    keepspaces=true,                 
    numbers=left,                    
    numbersep=5pt,                  
    showspaces=false,                
    showstringspaces=false,
    showtabs=false,                  
    tabsize=2
}

\lstset{style=mystyle}

\title{Relazione Progetto Sistemi Operativi}
\author{Ferrando Damillano Filippo, Nicosia Francesco, Noto Nicola}
\date{Gennaio 2024}

\begin{document}
\maketitle

\tableofcontents

\section{Membri del Gruppo}

\textbf{Ferrando Damillano Filippo}
\begin{itemize}
    \item Matricola: 1043397
    \item Turno: T1    
    \item Email: \href{mailto:filippo.ferrandodami@edu.unito.it}{filippo.ferrandodami@edu.unito.it}
\end{itemize}
\textbf{Nicosia Francesco}
\begin{itemize}
    \item Matricola: 1030308
    \item Turno: T4
    \item Email: \href{mailto:francesco.nicosia@edu.unito.it}{francesco.nicosia@edu.unito.it}
\end{itemize}
\textbf{Noto Nicola}
\begin{itemize}
    \item Matricola: 1055839
    \item Turno: T3
    \item Email: \href{mailto:nicola.noto@edu.unito.it}{nicola.noto@edu.unito.it}
\end{itemize}

\section{Presentazione del Progetto}
Il Progetto presentato nell'anno 23/24 richiede la simulazione di una reazione a catena di atomi.
Questa reazione a catena è gestita attraverso l'utilizzo di processi differenti per alimentare e allo stesso tempo mantenere sotto controllo la reazione (questi processi verranno visti in dettaglio nelle sezioni successive).


\section{Settaggio Variabili per l'Esecuzione}
Per il corretto utilizzo della simulazione, questa ha bisogno di alcune variabili che sono definite dall'utente prima della partenza della stessa.
Il metodo che abbiamo scelto sfrutta le variabili d'ambiente di Linux, abbiamo infatti untilizzato lo script
\begin{lstlisting}
    env-setter.sh
\end{lstlisting}
per definire tutte le variabili necessarie, queste sono raggiunte dal processo Master attraverso la chiama alla funzione
\begin{lstlisting}
    getenv()
\end{lstlisting}

\section{Processo Master}

\section{Processo Atomo}

\section{Processo Alimentatore}

\section{Processo Attivatore}

\section{Processo Inibitore}


\section{Soluzione Attuata}
Soluzione finale utilizzata

\section{Note Finali}


\section{Link al Progetto}
    \begin{itemize}
        \item \href{https://github.com/filippo-ferrando/progetto-so}{Github}
        \item \href{https://github.com/filippo-ferrando/progetto-so/archive/refs/heads/main.zip}{Zip File Download}
    \end{itemize}

\end{document}